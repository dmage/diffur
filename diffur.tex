\documentclass[a4paper,10pt]{report}
\usepackage[utf8]{inputenc}
\usepackage[english,russian]{babel}
\usepackage[colorlinks]{hyperref}
\usepackage{graphics}
\usepackage{wrapfig}
\usepackage{amsmath}
\usepackage{amsfonts}
\usepackage{amssymb}
\usepackage{amsthm}
\usepackage{makeidx}

% \DeclareGraphicsRule{*}{eps}{*}{}

\newcommand{\bbC}{\mathbb{C}}
\newcommand{\bbN}{\mathbb{N}}
\newcommand{\bbQ}{\mathbb{Q}}
\newcommand{\bbR}{\mathbb{R}}
\newcommand{\bbZ}{\mathbb{Z}}
\newcommand{\ud}{\mathrm{d}}

\newcommand{\new}[2]{\emph{#1}\index{#2}}
\newcommand{\important}[1]{\textbf{#1}}
\newcommand{\ednote}{(\textit{прим.\ ред.})}

\newtheorem{theorem}{Теорема}[section]
\newtheorem{identity}[theorem]{Критерий}
\theoremstyle{definition}
\newtheorem{definition}{Определение}[section]
\theoremstyle{remark}
\newtheorem{note}{Замечание}[section]
\newtheorem{example}{Пример}[section]
\theoremstyle{plain}
\newtheorem{lemma}[theorem]{Лемма}
\newtheorem{proposition}[theorem]{Предложение}
\newtheorem{corollary}{Следствие}[theorem]

\DeclareMathOperator{\sgn}{sgn}

\makeindex

\begin{document}

\tableofcontents

\chapter{Курс дифференциальных уравнений}

\section{Основные определения}
\begin{definition}[уравнение 1-го порядка]
 Соотношение вида
 \begin{equation}
  y^{(n)} = f\left(x, y, y', y'', \ldots, y^{(n-1)}\right) \tag{*}
  \label{eq:1_star}
 \end{equation}
 называется \new{обыкновенным дифференциальным уравнением $n$-го порядка}{Обыкновенное дифференциальное уравнение} разрешенным относительно старших производных (или коротко: уравнение $n$-го порядка), т.е.\ порядок уравнения~--- порядок старшей производной.
\end{definition}

\begin{note}
 Можно рассматривать более общее уравнение
 \[
  F\left(x, y, y', y'', \ldots, y^{(n-1)}, y^{(n)}\right) = 0.
 \]
\end{note}

\begin{definition}[решение]
 \new{Решением уравнения}{Решение уравнения} \eqref{eq:1_star} на множестве $E$ называются функции $y = \phi(x)$ имеющие на $E$ производные до $n$-го порядка включительно, которые будучи подставленными в уравнение \eqref{eq:1_star} превращают его в тождество.
\end{definition}

Рассмотрим частный случай, когда $n = 1$:
\begin{equation}
 y' = f(x,y) \qquad \text{или} \qquad \frac{\ud y}{\ud x} = f(x,y).
 \label{eq:1_1}
\end{equation}

Это уравнение допускает естественную геометрическую интерпретацию (рисунок \ref{fig:1_1}).

\begin{wrapfigure}{R}{.7\textwidth}
 \begin{center}
  \includegraphics{diffur/1}
  % \includegraphics[width=0.48\textwidth]{gull}
 \end{center}
 \caption{Здесь должна быть зарисовка.}
 \label{fig:1_1}
\end{wrapfigure}

Пусть $f(x,y)$ определена в области $g$ и пусть $M_0(x_0,y_0) \in g$. Рассмотрим решение $y = \phi(x)$ проходящее через точку $M_0$ (в предположении что такое выражение существует). Тогда $\phi'(x) = f(x, \phi(x))$ и при этом $\phi(x_0) = y_0$ FIXME\footnote{FIXME! Здесь $\phi$ или $\phi'$?}.

Таким образом для любой точки $M_0(x_0,y_0)$ $\tg \alpha = f(x_0,y_0)$.

Таким образом само уравнение \eqref{eq:1_1} в каждой точке области $g$ задает направление касательной к решению проходящему через эту точку FIXME\footnote{FIXME! \emph{нет текста}}.

Сопоставим каждой точке в области $g$ отрезок показывающий направление касательной к решению проходящему через эту точку. Полученная картинка носит название \new{поля направлений}{Поле направлений} определяемым уравнением \eqref{eq:1_1} в области $g$.

Таким образом само уравнение \eqref{eq:1_1} в области $g$ задает поле направлений.

\begin{note}
 Изложенная геометрическая интерпретация имеет существенные недостатки, а именно: заведомо исключаются из рассмотрения вертикальные направления и тем самым исключаются из рассмотрения такие важные объекты как циклы и спирали.
\end{note}

Расширим нашу геометрическую интерпретацию следующим образом: рассмотрим на ряду с уравнением \eqref{eq:1_1} так называемое <<перевернутое уравнение>>
\begin{equation}
 \frac{\ud x}{\ud y} = \frac{1}{f(x,y)}. \tag{\ref{eq:1_1}'}
 \label{eq:1_1a}
\end{equation}

Те точки области $g$, в которых не определен угловой коэффициент к оси $Ox$, но определен по отношению к оси $Oy$ присоединим к области $g$, т.е.\ рассмотрим более широкую область $g_1 \supseteq g$. Тем самым в области $g_1$ работает хотя бы одно уравнение из \eqref{eq:1_1} и \eqref{eq:1_1a}.

\begin{definition}[интегральная кривая]
 Кривая, имеющая в каждой своей точке уравнение касательной определенной в виде \eqref{eq:1_1} или \eqref{eq:1_1a} называется \new{интегральной кривой}{Интегральная кривая} уравнения \eqref{eq:1_1}.
\end{definition}

\begin{note}
 При этом предполагается что правые части уравнения \eqref{eq:1_1}, допускающие в каждой точке до определение по непрерывности, до определены по непрерывности.
 \begin{example}
  $$\frac{\ud y}{\ud x} = \frac{x^2+y^2}{x^2+y^2}.$$
 \end{example}
\end{note}

\begin{definition}[интеграл уравнения \eqref{eq:1_1}]
 Уравнение $\Phi(x,y) = 0$ FIXME\footnote{FIXME! \emph{нет текста}} называется \new{интегралом уравнения}{Интеграл уравнения} \eqref{eq:1_1}.
\end{definition}

Таким образом интеграл~--- обобщение понятия решения, а интегральная кривая~--- понятия графика решения.

\begin{example}
 FIXME!\footnote{FIXME! Довольно мерзкое форматирование с графиками\ldots}
\end{example}

Запишем уравнение \eqref{eq:1_1} в симметричной форме:
\begin{equation}
 \frac{\ud y}{\ud x} = \frac{Q(x,y)}{P(x,y)}
 \label{eq:1_2}
\end{equation}
и
\begin{equation}
 \frac{\ud x}{\ud y} = \frac{P(x,y)}{Q(x,y)}. \tag{\ref{eq:1_2}'}
 \label{eq:1_2a}
\end{equation}

Забегая вперед: уравнение \eqref{eq:1_2}~--- дифференциальное уравнение фазовой траектории следующей автономной системы: FIXME\footnote{FIXME! Все верно?}
\begin{equation}
 \left\{\begin{array}{l}
  \dfrac{\ud x}{\ud t} = P(x,y) \\
  \dfrac{\ud y}{\ud t} = Q(x,y)
 \end{array}\right.
\end{equation}

\begin{flalign*}
 \frac{\partial\Phi}{\partial x} + \frac{\partial\Phi}{\partial y} & = 0 \\
 P \ud y & = Q \ud x \\
 \frac{\ud\Phi}{\ud x} P \ud x + \frac{\ud\Phi}{\ud y} P \ud y & = 0 \\
 \left(\frac{\partial\Phi}{\partial x} + \frac{\partial\Phi}{\partial y}\right) \ud x & = 0 \\
 \Longrightarrow \frac{\partial\Phi}{\partial x} + \frac{\partial\Phi}{\partial y} & = 0.
\end{flalign*}

При весьма общих условиях на функциях $P$ и $Q$ можно доказать что соотношение
\[
 \frac{\partial\Phi}{\partial x} + \frac{\partial\Phi}{\partial y} = 0
\]
является необходимым и достаточным для того, чтобы уравнение
\[
 \Phi(x,y) = 0 
\]
было интегралом нашего уравнения \eqref{eq:1_2}.

Этот результат будет получен дальше как тривиальное частное следствие теоремы о первых интегралах автономной системы.

\begin{definition}[общее решение]
 Однопараметрическое семейство решение $y = \phi(x, C)$ называется \new{общим решением}{Общее решение} уравнения в области $g$ если при надлежащем выборе параметра $C$ оно даст любое решение нашего уравнения \eqref{eq:1_1}, график которого лежит в области $g$.

 Иными словами, $y = \phi(x, C)$ есть общее решение уравнения \eqref{eq:1_1}, если для любого решения $\overline{\phi}(x)$, график которого лежит в области $g$, существует $\overline{C}$ такое, что $\phi(x, \overline{C}) \equiv \overline{\phi}(x)$.
 \label{def:1_5}
\end{definition}

\begin{definition}[общий интеграл уравнения \eqref{eq:1_1}]
 Соотношение
 \[
  \Phi(x,y,C)
 \]
 называется \new{общим интегралом уравнения}{Общий интеграл уравнения} \eqref{eq:1_1} в области $g_1$, если при надлежащем выборе параметра $C$ оно дает любую интегральную кривую уравнения \eqref{eq:1_1} лежащую в области $g_1$.
 \label{def:1_6}
\end{definition}

В определении \ref{def:1_5} и \ref{def:1_6} исключительную роль играет область, т.к.\ одно и тоже решение в одной области может быть общим решением, а в другой вовсе не быть таковым.

\begin{example}
  FIXME!\footnote{FIXME! Довольно мерзкое форматирование с графиками\ldots}
\end{example}

\begin{wrapfigure}{r}{.4\textwidth}
 \begin{center}
  \includegraphics{diffur/2}
 \end{center}
 \caption{FIXME! Рисунок напротив задачи Коши.}
\end{wrapfigure}

\begin{definition}[задача Коши]
 Совокупность дифференциального уравнения \eqref{eq:1_1} и начального условия $y(x_0) = y_0$ называется \new{задачей Коши}{Задача Коши} для уравнения \eqref{eq:1_1}.
\end{definition}

Геометрической интерпретация двоякая:
\begin{enumerate}
 \item в узком смысле\\
  Найти решение дифференциального уравнения \eqref{eq:1_1} проходящего через точку $(x_0,y_0)$.
 \item в широком смысле\\
  Найти интегральную кривую уравнения \eqref{eq:1_1} проходящую через точку $(x_0,y_0)$.
\end{enumerate}

\begin{definition}[особая точка]
 Точка $M_0(x_0,y_0)$ называется \new{особой точкой}{Особая точка} дифференциального уравнения \eqref{eq:1_1}, если выполняется два условия:
 \begin{enumerate}
  \item В самой точке $M_0$ поле направлений не определено ни уравнение \eqref{eq:1_1}, ни уравнение \eqref{eq:1_1a},
  \item Существует $\dot{O}_\delta(M_0)$ в которой поле направлений определено либо уравнением \eqref{eq:1_1}, либо уравнением \eqref{eq:1_1a}.
 \end{enumerate}
\end{definition}

\begin{example}
  FIXME!\footnote{FIXME! Довольно мерзкое форматирование с графиками\ldots}
\end{example}

\begin{note}
 Рассмотрим уравнение \eqref{eq:1_2}. Очевидно, что для того, чтобы $M_0(x_0,y_0)$ была особой точкой уравнения \eqref{eq:1_2} необходимо, чтобы $P(x_0,y_0) = Q(x_0,y_0) = 0$.

 Это условие не является достаточным для того, чтобы точка являлась особой.
\end{note}

\begin{definition}[цикл]
 \new{Циклом}{Цикл} уравнения \eqref{eq:1_1} называется замкнутая интегральная кривая этого уравнения не проходящая через его особые точки.
\end{definition}

\begin{definition}[исключительное направление]
 Направление, по которому хотя бы 1 интегральная кривая ??? ?? особой точки уравнения \eqref{eq:1_1} называется исключительным направлением FIXME\footnote{Пропущен текст}
\end{definition}

\begin{example}
 Само исключительное направление вовсе не обязано быть интегральной прямой уравнения \eqref{eq:1_1}, но в частности может ею быть.
\end{example}

\begin{definition}[изоклина]
 Линия, вдоль которой поле направлений уравнения \eqref{eq:1_1} постоянно, называется \new{изоклиной}{Изоклина} этого уравнения.
\end{definition}

\begin{example}
 \[
  \frac{\ud y}{\ud x} = \frac{Q(x,y)}{P(x,y)}
 \]
 \begin{enumerate}
 \item $Q(x,y) = 0$~--- изоклина нуля,
 \item $P(x,y) = 0$~--- изоклина бесконечности,
 \item $P(x,y)-Q(x,y) = 0$~--- изоклина 1.
\end{enumerate}
\end{example}

\begin{definition}[сепаратриса]
 Сепаратрисой уравнения \eqref{eq:1_1} относящейся к данной особой точке называется интегральная кривая этого уравнения удовлетворяющая двум условиями:
 \begin{enumerate}
 \item Сама эта кривая проходит через данную особую точку.
 \item Сколько угодно близко к этой интегральной кривой существует интегральные кривые уравнения \eqref{eq:1_1} не проходящие через данную особую точку.
\end{enumerate}
\end{definition}

\begin{example}[сепаратриса]
 \mbox{}
 \begin{enumerate}
  \item
  \[
   y' = \frac{x}{y}
  \]
  \begin{center}
   \includegraphics{diffur/3}
  \end{center}
  Здесь сепаратриса состоит из 2-х пар усов.

  \item
  \[
   y' = \frac{|x|}{y} = \begin{cases}
    \frac{x}{y}, & x \ge 0 \\
    -\frac{x}{y}, & x < 0
   \end{cases}
  \]
  \begin{center}
   \includegraphics{diffur/4}
  \end{center}
  Здесь сепаратриса состоит из 1-й пары усов.

  \item
  \[
   \frac{\ud y}{\ud x} = \frac{x^2}{y} \qquad (0,0)
  \]

  FIXME\footnote{FIXME! Мерзкое форматирование ну и т.д.}

  \item
  \[
   y' = \frac{2x^3}{y}
  \]

  \item
  \[
   y' = \frac{-x+x^3}{y}
  \]

  \item
  \[
   y' = \frac{x-x^3}{y}
  \]
 \end{enumerate}
\end{example}

\begin{definition}[нарушение единтсвенности]
 Мы скажем, что в точке $M_0(x_0,y_0)$ происходит нарушение единственности уравнения \eqref{eq:1_1}, если через эту точку по данному направлению, определяемому либо уравнением \eqref{eq:1_1}, либо уравнением \eqref{eq:1_1a}, проходят 2 интегральные кривые не совпадающие между собой в сколь угодно малой окрестности точки $M_0$.
\end{definition}

\begin{note}
 В частности, из этого определения следует, что в особой точке единственность никогда не нарушается.
\end{note}

\begin{example}
 FIXME\footnote{FIXME!}
\end{example}

\begin{definition}[неограниченная продолжаемость]
 Решение уравнения \eqref{eq:1_1} проходящее через $M_0(x_0,y_0$ называется
 \begin{enumerate}
  \item неограниченно продолжаемым вправо, если оно определено на полуоси $[x_0, +\infty)$,
  \item неограниченно продолжаемым влево, если оно определено на полуоси $(-\infty, x_0]$,
  \item двусторонняя неограниченная продолжаемость, если оно определено на $(-\infty, +\infty)$.
 \end{enumerate}
 В противном случае говорят, что наше решение имеет конечное время определения.
\end{definition}

\begin{example}
 FIXME\footnote{FIXME!}
\end{example}

\section{Уравнения с разделяющимися переменными}
\begin{equation*}
 \frac{\ud y}{\ud x} = f(x) g(y)
\end{equation*}

\subsection{Уравнение вида $\displaystyle\frac{\ud y}{\ud x} = f(x)$}
\begin{theorem}[о существовании и единственности решения]
 Пусть функция $f(x) \in C(a,b)$. Тогда через каждую точку $M_0(x_0,y_0) \in G = \{a < x < b; -\infty < y < +\infty\}$ проходит, и при том только одно, решение уравнения
 \begin{equation}
  \frac{\ud y}{\ud x} = f(x)
  \label{eq:2_1}
 \end{equation}
\end{theorem}
\begin{proof}[Доказательство существования]
 \[
  \exists: y = y_0 + \int_{x_0}^x f(t)\; \ud t
 \]

 $y = f(x)$ (согласно теореме Ньютона-Лейбница).

 FIXME\footnote{Доказательство несколько тривиально и нуждается в дополнении.}
\end{proof}
\begin{proof}[Доказательство единственности]
 Допустим, что существует 2 решения
 \[
  \varphi_1(x) \qquad \text{и} \qquad \varphi_2(x),
 \]
 т.е.
 \begin{gather*}
  \varphi_1'(x) = f(x) \qquad \text{и} \qquad \varphi_1(x_0) = y_0, \\
  \varphi_2'(x) = f(x) \qquad \text{и} \qquad \varphi_2(x_0) = y_0.
 \end{gather*}

 Рассмотрим разность
 \[
  \varphi(x) = \varphi_1(x) - \varphi_2(x).
 \]
 Тогда ее производная будет равна
 \[
  \varphi'(x) = \varphi_1'(x) - \varphi_2'(x) = 0,
 \]
 т.е. $\varphi(x)$ является константой.

 \[
  \varphi_1(x_0) - \varphi_2(x_0) = 0 \Longrightarrow \varphi(x_0) = 0.
 \]

 Из всего выше сказанного следует, что
 \[
  \varphi(x) = 0.
 \]
\end{proof}
\begin{note}
 Условие непрерывности функции $f$ существенно для существования решения.

 Рассмотрим
 \begin{gather*}
  y' = f(x) \qquad f(x) = \sgn(x) \qquad (0,0) \\
  y = |x| + C
 \end{gather*}
\end{note}
\begin{note}
 Тем не менее отнюдь не является необходимым для существования решения
 FIXME\footnote{FIXME!}
\end{note}
\begin{note}
 Оказывается, что для единственности решения <совершенно не необходимо условие FIXME\footnote{FIXME!}>
\end{note}
\begin{theorem}
 Пусть функция f(x) определена на $(a,b)$. Тогда через любую точку $M_0(x_0,y_0)$ проходит не более одного решения уравнения \eqref{eq:2_1}.
\end{theorem}
\begin{proof}
 смотри пункт 2 доказательства теоремы 1.
\end{proof}
\begin{note}
 Не трудно убедится, что при условии теоремы 1 формула
 \[
  y = C + \int_{x_0}^x f(t)\; \ud t
 \]
 представляет собой общее решение уравнения \eqref{eq:2_1} в области $G$.
\end{note}

\subsection{Поведение решений уравнения $\displaystyle\frac{\ud y}{\ud x} = f(x)$ в случае, когда функция $f(x)$ имеет точку бесконечного разрыва}
Пусть $f(x) \in C\big((a,c) \cup (c,b)\big)$ и $\displaystyle\lim_{x \to c} |f(x)| = +\infty$. Тогда
\begin{equation}
 \frac{\ud x}{\ud y} = \begin{cases}
  \frac{1}{f(x)}, & x \ne c \\
  0, & x = c.
 \end{cases}
 \label{eq:2_dxdy}
\end{equation}

Следовательно $x \equiv c$ является решением уравнения \eqref{eq:2_dxdy}, т.е. это интегральная прямая уравнения \eqref{eq:2_1}.

FIXME\footnote{РИСУНОК!}

Пусть точка $M_0(x_0,y_0) \in G_1$, тогда согласно теореме 1 FIXME\footnote{FIXME! Нужно использовать \LaTeX нумерацию.} через точку $M_0(x_0,y_0)$ проходит единственное решение уравнений \eqref{eq:2_1}.

\[
 y = y_0 + \int_{x_0}^x f(t)\; \ud t_x
\]

Пусть $x \in c-0$, тогда $\int_{x_0}^x f(t)\; \ud t \to I_1 = \int_{x_0}^c f(t)\; \ud t$, где $a < x_0 < c$.

Если интеграл $I_1$ расходится, то вертикальная примая $c$ служит асимптотой. В случае же, если интеграл сходится, то интегральная прямая касается прямой $c$, т.е. произойдет нарушение единственности в этой точке.

\rule{.3\textwidth}{.5pt} 14.09.2007\\

Поведение нашей интегральной кривой уравнения (1) в области $G_1$ определяется несобственным интеграллом
\[
 I_1 = \int_{x_0}^c f(t)\; \ud t \qquad a < x_0 < c.
\]

Очевидно, что все остальные интегральный кривые ведут себя точно так же.

Аналогичное рассуждение можно провести для области $G_2$, при это естественно возникает, что несобственный интеграл
\[
 I_2 = \int_{x_0}^c f(t)\; \ud t \qquad c < x_0 < b.
\]
FIXME\footnote{Упражнение: Провести эти рассуждения.}

Заключение: Таким образом поведение интегральных кривых уравнения 1 $y' = f(x)$ в окрестности вертикальной интегральной кривой зависит от сходимости или расходимости несобственных интегралов $I_1$ и $I_2$ и знаков этих самых пределов\footnote{Здесь подразумеваются пределы из определения несобственных интегралов. \ednote}.

Таким образом возможны следующие случаи:
\begin{enumerate}
 \item $\displaystyle\lim_{x \to c} f(x) = +\infty,$
 \item $\displaystyle\lim_{x \to c} f(x) = -\infty,$
 \item $\displaystyle\lim_{x \to c-0} f(x) = +\infty \quad \text{и} \quad \lim_{x \to c+0} f(x) = -\infty,$
 \item $\displaystyle\lim_{x \to c-0} f(x) = -\infty \quad \text{и} \quad \lim_{x \to c+0} f(x) = +\infty,$
\end{enumerate}
\begin{enumerate}
 \renewcommand{\theenumi}{\Asbuk{enumi}}
 \item интегралы $I_1$ и $I_2$ расходятся,
 \item интегралы $I_1$ и $I_2$ сходятся,
 \item интеграл $I_1$ расходится, а интеграл $I_2$ сходится.
 \item интеграл $I_1$ сходится, а интеграл $I_2$ расходится.
\end{enumerate}
Комбинируя предложения 1-4 и А-Г  мы получим 16 различных случаев поведения интегральных кривых нашего уравнения в окрестности вертикальной кривой.

\begin{example}
 FIXME
 Здесь рассматривается случай 1 Б.

 \begin{equation*}
  \frac{\ud y}{\ud x} = \frac{1}{\sqrt{|x-1|}}
 \end{equation*}
\end{example}

\begin{note}
 Из всего этого вытекает, что единственность решения в точке $x = c$ нарушается тогда и только тогда, когда сходится сходится хотя бы один из двух интегралов.
\end{note}
\begin{note}
 В точках не лежащих на интегральной прямой $x = c$ единственность решения не нарушается никогда, даже если через эту точку проходит бесконечное множество решений.

 Единственность не нарушается, т.к. все эти решения различаются лишь глобально. Локально же они совпадают.
\end{note}

\subsection{Уравнение вида $\displaystyle\frac{\ud y}{\ud x} = g(y)$}
\begin{equation}
 \frac{\ud y}{\ud x} = g(y)
 \label{eq:dy/dx=g(y)}
\end{equation}

Имеет место теорема (2) (существования и единственности решения этого уравнения):
\begin{theorem}
 Пусть выполнены условия
 \begin{enumerate}
  \item $g(y) \in C(a,b),$
  \item $g(y) \ne 0 \forall y \in (a,b).$
 \end{enumerate}
 Тогда через каждую точку $M_0(x_0,y_0) \in D = {-\infty < x < +\infty, a < y < b}$ проходит, и при том только одно, решение нашего уравнения.
\end{theorem}
\begin{proof}
 Из второго условия вытекает, что уравнение \eqref{eq:dy/dx=g(y)} и уравнение
 \begin{equation}
  \frac{\ud x}{\ud y} = \frac{1}{g(y)}
  \tag{\ref*{eq:dy/dx=g(y)}$'$}
  \label{eq:dx/dy=1/g(y)}
 \end{equation}
 равносильны.

 При этом получается, что $\frac{1}{g(x)} \in C(a,b)$ (по теореме непрерывности част\ldots FIXME\footnote{FIXME! Что он тут сказал?!}) FIXME\footnote{FIXME! И тут опять что-то сказал невнятное.} к уравнению \eqref{eq:dx/dy=1/g(y)} (см. 1 из I). Следовательно к этому уравнению применима теорема I, согласно которой через точку $M_0(x_0,y_0)$ проходит, и при том только одно, решение \eqref{eq:dx/dy=1/g(y)}
 \[
  x = x_0 + \int_{y_0}^y \frac{\ud t}{g(t)}.
 \]
 Введем вспомогательную функцию
 \[
  G(y) = \int_{y_0}^{y} \frac{\ud t}{g(t)}.
 \]
 Тогда можно записать
 \[
  G(y) = x-x_0.
 \]

 STY-FIXME-FWD
 \[
  g'(y) = \frac{1}{g(y)} \ne 0
 \]
 $\Longrightarrow g(y)$ монотонная функция и следовательно имеет однозначную обратную функцию $G^{-1}(y)$.

 \[
  G^{-1}(G(y)) = g
 \]

 \begin{equation}
  y = G^{-1}(x-x_0)
 \end{equation}

 Уравнение 3 есть решение.

 Единственность решения гарантируется теоремой (1) примененной к \eqref{eq:dx/dy=1/g(y)}.
\end{proof}
\begin{note}
 Условие непрерывности функции $g(y)$ существенно для существования решения.
\end{note}
\begin{example}
 \begin{equation*}
  \frac{\ud y}{\ud x} = g(y) = \begin{cases}
   1, & y > 0 \\
   2, & y = 0 \\
   -1, & y < 0.
  \end{cases}
 \end{equation*}

 Через точку $(0,0)$ не проходит ни одного решения нашего уравнения.

 FIXME\footnote{FIXME! Здесь должна быть зарисовка, однако я еще не понял, должны ли выходить усы из $(0,0)$ или нет. Если нет, то в чем разница относительно следующего примера?}
\end{example}
\begin{note}
 Тем не менее, условие непрерывности отнюдь не является необходимым.
\end{note}
\begin{example}
 \begin{equation*}
  \frac{\ud y}{\ud x} = g(y) = \sgn(y).
 \end{equation*}

 FIXME\footnote{FIXME! См.\ предыдущее замечание}
\end{example}
\begin{note}
 Легко показать, что условие $g(y) \ne 0$
 \begin{enumerate}
  \renewcommand{\theenumi}{\asbuk{enumi}}
  \item существенно для единственности решения,
  \item не является необходимым.
 \end{enumerate}
 Соответствующие примеры будут даны в IV пункте.
\end{note}
\begin{note}
 Легко показать, что при условиях Т2 формула
 \[
  G(y) = x-C
 \]
 представляет собой общий интеграл в области $D$, а формула
 \[
  y = G^{-1}(x-C)
 \]
 представляет собой общее решение уравнения \eqref{eq:dy/dx=g(y)} в области $D$.
\end{note}

\subsection{Поведение интегральных кривых уравнения $\displaystyle\frac{\ud y}{\ud x} = g(y)$ в случае, когда функция $g(y)$ обращается в нуль в отдельных точках.}
Пусть
\begin{enumerate}
 \item $g(y) \in C(a,b)$
 \item $g(c) = 0$
 \item $g(y) \ne 0 \forall y \ne c$
\end{enumerate}
Тогда $y \equiv c$ есть решение уравнения 1.

РИСУНОК!

В областях $D_1$ и $D_2$ работает теорема (2), согласно которой через точку $M_0(x_0,y_0)$ проходит единственная интегральная кривая по\footnote{Что-то тут не так, нужно исправить \ednote} формуле
\begin{equation}
 x = x_0 + \int_{y_0}^y \frac{\ud t}{g(t)}.
\end{equation}

Устремим $y$ к точке $c$ слева. $\Longrightarrow \int_{y_0}^y \frac{\ud t}{g(t)} \to \int_{y_0}^c \frac{\ud t}{g(t)} \quad a < y_0 < c$

В случае, если интеграл $I_1$ расходится, наше решение $I_1$ будет служить горизонтальной асимптотой для любых решений. Если же сходится, то наша интегральная кривая в какой-то точке касается $y = c$ и здесь нарушается единственность решения.

УПРАЖНЕНИЕ: Проделать подобные рассуждения для $D_2$.

Таким образом наше поведение интегральных прямых(кривых?) определяется следующими факторами:
сходимость и расходимость интегралов $I_1$, $I_2$ и \\
комбинациями знаков.

Тем самым мы получим 16 различных случаев поведения интегральных кривых уравнения 1 в нашей ...................................., которые получаются из аналогичных 16 случаев поведения интегральных кривых рассмотренных в пункте II с помощью поворота на $90^\circ$

УПРАЖНЕНИЕ: Нарисовать все эти 16 интегральных кривых и проиллюстрировать их соответствующими примерами.

\subsection{Продолжаемость уравнения (1) в полуплоскости или во всей плоскости}
Рассмотрим 2 различные задачи Коши.

FIXME(SKIPPED)

Рассмотрим вопрос о неограниченной продолжаемости уравнения 1

Пусть
\begin{enumerate}
 \item $g(y) \in C(-\infty,+\infty)$
 \item $g(y) \ne 0 \forall y$
\end{enumerate}
Тогда через точку $M_0(x_0,y_0)$ проходят интегральные кривые 2
\[
 x = x_0 + \int_{y_0}^y \frac{\ud t}{g(t)}
\]

/// Для огр g(y) > 0\\
Допустим, что наше решение неограниченно продолжаемо вправо, т.е. определено на полуоси $(x_0,+\infty)$.

Устремим $x$ к $+\infty$. (Как при этом себя будет вести $y$?) $у$ возрастает.

Есть 2 возможности:
FIXME

Допустим, что предел $\lim_{x \to +\infty} y(x) = \overline{y}$ FIXME

Остается возможность $x \to +\infty$ $y (x) = +\infty$.

Естественно возникает $K_1 = \int_y^{+\infty} \frac{\ud t}{g(t)}$, причем $K_1 = +\infty$.

Если решение неограниченно продолжаемое вправо, то несобственный интеграл $K_1$ расходится к $+\infty$. Оказывается, что верно и обратное утверждение:

\begin{lemma}[\footnote{лектор это не выделял как теорему!!}]
 Если несобственный интеграл
 \[
  K_1 = \int_y^{+\infty} \frac{\ud t}{g(t)}
 \]
 расходится к $+\infty$, то наше решение неограниченно продолжаемо вправо.
\end{lemma}
\begin{proof}
 Допустим противное, т.е. существует $\overline{x}$ являющееся решением определенном на $(x_0,\overline{x})$

 \[
  \lim_{x \to \overline{x}-0} y(x) = +\infty
 \]

 A это нас приводит к противоречию.
\end{proof}

Вывод: При сделанных предположениях для того, что любое решение уравнения 1 было неограниченно продолжаемо вправо необходимо и достаточно, чтобы интеграл $K_1$ расходился.

FIXME\footnote{Упражнения}









\rule{.3\textwidth}{.5pt} \\

\begin{note}
 Из доказанного критерия вытекает, что, в случае когда $g(y) \ne 0$, сосуществование неограниченно продолжаемых решений и решений с конечным временем определения невозможно.
\end{note}
\begin{example}
 \[
  \frac{\ud y}{\ud x} = y^2 - 1
 \]

 \[
  \int_{y_0 > 1}^y \frac{\ud t}{t^2-1}
 \]

 Интеграл $\int_0^{+\infty} \frac{\ud t}{t^2}$ сходится.\footnote{Хотелось бы поподробнее рассмотреть пример.}
\end{example}

\subsection{Уравнение вида $\displaystyle \frac{\ud y}{\ud x} = f(x) \cdot g(y)$}
\begin{equation}
 \frac{\ud y}{\ud x} = f(x) \cdot g(y)
 \label{eq:dy/dx=f(x)g(y)}
\end{equation}

\begin{theorem}[существование и единственность решения уравнения с разделяющимися переменными]
 Пусть выполнены условия:
 \begin{enumerate}
  \item $\displaystyle f(x) \in C(a,b) \qquad a, b \in \left[-\infty,+\infty\right]$,
  \item $\displaystyle g(x) \in C(c,d) \qquad c, d \in \left[-\infty,+\infty\right]$,
  \item $g(y) \ne 0 \quad \forall y \in (c,d)$.
 \end{enumerate}
 Тогда через любую точку $M_0(x_0,y_0)$ из множества
 \[
  D = \left\{
   a < x < b,
   c < y < d
  \right\}
 \]
 проходит единственное решение уравнения \eqref{eq:dy/dx=f(x)g(y)}.
\end{theorem}
\begin{proof}[Доказательство единственности]
 Допустим, что существует решение $y = \varphi(x)$
 уравнения \eqref{eq:dy/dx=f(x)g(y)}, проходящее через точку $M_0(x_0,y_0)$, т.е. выполняется равенство
 \[
  \varphi(x_0) = y_0.
 \]
 Тогда для него будет верно равенство
 \[
  \varphi'(x) = f(x) \cdot g(\varphi(x))
 \]
 или
 \[
  \frac{\varphi'(x)}{g(\varphi(x))} = f(x).
 \]
 Проинтегрировав, получаем
 \[
  \int_{x_0}^x f(t)\; \ud t
  = \int_{x_0}^x \frac{\varphi'(x)}{g(\varphi(x))}\; \ud t
  = \int_{x_0}^x \frac{\ud \varphi(x)}{g(\varphi(x))}.
 \]

 Введем в рассмотрение вспомогательные функции
 \[
  F(x) = \int_{x_0}^x f(t)\; \ud t
  \qquad \text{и} \qquad
  G(y) = \int_{y_0}^y \frac{\ud t}{g(t)}.
 \]

 Используя выше полученный результат получаем
 \[
  F(x) = G(\varphi(x)) - G(\varphi(x_0)) = G(\varphi(x)).
 \]

 Найдем производную
 \[
  G'(y) = \frac{1}{g(y)} \ne 0,
 \]
 следовательно функция $G(y)$ монотонна и существует функция $G^{-1}$, обратная к $G(y)$, для которой выполняется равенство
 \[
  G^{-1}(G(y)) = y.
 \]

 Таким образом, мы получили, что
 \[
  \varphi(x) = G^{-1}(F(x)).
 \]
\end{proof}
\begin{proof}[Доказательство существования]
 Рассмотримм функцию, определенную формулой (4)\footnote{Это где? :)}
 \begin{flalign*}
  y & = \varphi(x) = G^{-1}(F(x)) \\
  y(x_0) & = \varphi(x_0) = G^{-1}(F(x_0)) = y_0 \\
  G(\varphi(x)) & = F(x) \\
  F'(x) & = G'(\varphi(x)) \\
  f(x) & = \frac{1}{g(\varphi(x))} \cdot \varphi'(x) \\
  \varphi'(x) & = f(x) \cdot g(\varphi(x)).
 \end{flalign*}
\end{proof}
\begin{note}
 Условие непрерывности функций $g(y)$ и $f(x)$ является необходимымм для существования и единственности решения.
\end{note}
\begin{note}
 Если мы в формуле 
\end{note}



\emph{to be continued...}

\printindex

\end{document}
